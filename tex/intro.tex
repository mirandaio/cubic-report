\section{Introduction}
  The current mechanism for creating the probe stream in TCP Rapid increases 
  the sending rates of the packets in the probe stream exponentially. The
  mechanism for calculating the sending rates of the packets in a p-stream in 
  TCP Rapid is given by
  \begin{IEEEeqnarray}{rCl}
    r_0 & = & r_{avg} \\
    r_i & = & m^{i-1}*r_1, \qquad 1 < i < N
  \end{IEEEeqnarray}
  where
  \begin{equation}
    r_1 = \frac{m^{N-1} - 1}{(N-1)(m-1)m^{N-2}}r_{avg},
  \end{equation}
  $r_{avg}$ is the average sending rate of the p-stream, $m$ is the spread 
  factor, and $N$ is the number of packets in the probe stream.
  
  With this mechanism, an exponential range of sending rates is probed.
  Also, the gaps between the packets are first large and then decrease as more 
  packets are sent. That is, the granularity is coarser when the first 
  packets are being sent, and finer when the end of the p-stream is aproached. 
  This mechanism for assigning the rates of the packets in the p-stream would 
  work particularly well when the available bandwidth in the bottleneck link 
  is much larger than $r_{avg}$. This is because the granularity of the 
  p-stream is finer for packets with rates larger than $r_{avg}$, hence the 
  estimated available bandwidth can be calculated more accurately. The main 
  idea of this report is based on the insight provided by CUBIC TCP. It is a 
  good idea to assume that the new available bandwidth is not very far away 
  from the previous available bandwidth estimate. In this report, we describe a 
  mechanism for assigning the sending rates of the packets in the p-stream 
  so that the granularity is finer when close to the last estimate of the 
  available bandwidth.

